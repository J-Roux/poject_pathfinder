\section{Тестирование программы} 
\label{sec:program_design}

Целью тестирования является проверка работоспособности разработанного программного обеспечения. На этой стадии необходимо проверить корректное функционирование разработанного программного обеспечения и соответствие его требованиям, выдвинутым в техническом задании. При выявлении несоответствий работы программы техническому заданию, либо ошибок, требуется доработка программного обеспечения и (или) документации. Разработанное программное обеспечение должно гарантировать устойчивое функционирование независимо от действий конечных пользователей. При возникновении отдельных ошибочных ситуаций разработанное программное обеспечение должно их успешно обрабатывать. Для проведения корректного тестирования сначала необходимо разработать порядок испытаний. Порядок испытаний - это список последовательности действий направленных на проверку корректности работы программы и (или) ее отдельных функциональных частей.

Существует два основных вида тестирования: функциональное и структурное. При функциональном тестировании программа рассматривается как “черный ящик” (то есть ее текст не используется). Происходит проверка соответствия поведения программы ее внешней спецификации. Возможно ли при этом полное тестирование программы? Очевидно, что критерием полноты тестирования в этом случае являлся бы перебор всех возможных значений входных данных, что невыполнимо. 

Поскольку исчерпывающее функциональное тестирование невозможно, речь может идти о разработки методов, позволяющих подбирать тесты не “вслепую”, а с большой вероятностью обнаружения ошибок в программе. При структурном тестировании программа рассматривается как “белый ящик” (т.е. ее текст открыт для пользования). Происходит проверка логики программы. Полным тестированием в этом случае будет такое, которое приведет к перебору всех возможных путей на графе передач управления программы (ее управляющем графе). Даже для средних по сложности программ числом таких путей может достигать десятков тысяч. 

Таким образом, ни структурное, ни функциональное тестирование не может быть исчерпывающим. Рассмотрим подробнее основные этапы тестирования программных комплексов. В тестирование многомодульных программных комплексов можно выделить четыре этапа:
\begin{itemize}

\item тестирование отдельных модулей; 

\item совместное тестирование модулей; 

\item тестирование функций программного комплекса (т.е. поиск различий между разработанной программой и ее внешней спецификацией ); 

\item тестирование всего комплекса в целом (т.е. поиск несоответствия созданного программного продукта, сформулированным ранее целям проектирования, отраженным обычно в техническом задании). 
\end{itemize}

На первых двух этапах используются, прежде всего, методы структурного тестирования, т.к. на последующих этапах тестирования эти методы использовать сложнее из-за больших размеров проверяемого программного обеспечения; последующие этапы тестирования ориентированы на обнаружение ошибок различного типа, которые не обязательно связаны с логикой программы.

В процессе разработки использовалась отладочная версия приложения, которая параллельно с основным окном показывала консоль, на которую выводилась отладочная информация о работе системы, что использовалось при тестировании.

Тестирование модулей проводилось в процессе создания системы, когда данный модуль готов к использованию, при возможных ошибочных ситуациях информация выводилась на консоль, а затем исправлялось место в модуле, где возникала ошибка. 

Тестирование всего приложения проводилось  в результате многократных запусков, проходах по игровому циклу, разнообразных оправляемых событиях и моделируемых ошибочных ситуаций.

В результате тестирования была получено более защищенное приложения, адекватно реагирующее на возможные ошибочные ситуации.