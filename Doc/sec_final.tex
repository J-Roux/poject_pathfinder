\sectioncentered*{Заключение}
\addcontentsline{toc}{section}{Заключение}
\label{sec:final}


В ходе создания проекта было создано приложения для операционной системы Microsoft Windows, которое было написано в среде разработки Visual Studio 2012 с использованием библиотеки компьютерного зрения OpenCV.

В рамках изучения предметной области были рассмотрены основные методы обнаружения движений: метод межкадровой разности и метод вычитания фона. Также рассмотрены следующие методы обнаружения лиц: метод Виолы-Джонса и метод главных компонент.

В процессе работы с библиотекой OpenCV были использованы следующие её элементы: инструменты для работы с устройствами ввода и веб"=камерой, инструменты обработки изображений, алгоритмы компьютерного зрения, инструменты работы с окнами. Однако этим библиотека не ограничивается, и возможности её использования неизмеримы. Плюсом данной библиотеки является и то, что она является open sourсe продуктом --- это значит что любой программист, заинтересованный в развитии библиотеки может принять участие в этом.
 
В ходе тестирования были выявлены ошибки в проектировании системы и  места в программе, которые замедляют ее. Все данные недочеты были исправлены, что подтверждается в результате многочисленных тестовых запусках приложения.

Приложение можно использовать как визуализацию методов изученных ранее, и использовать как основу для домашней сигнализации и системы видеонаблюдения. Вместо дорогостоящих комплексных систем можно самому создать и использовать продукт, который решает конкретные задачи, поставленные перед системой видеонаблюдения.
