\sectioncentered*{Введение}
\addcontentsline{toc}{section}{Введение}
\label{sec:intro}

В настоящее время вычислительная техника используется во многих областях человеческой деятельности, являясь удобным и многофункциональным инструментом решения широкого круга задач. Многие отрасли техники, имеющие отношение к получению, обработке, хранению и передаче информации, в значительной степени ориентируются в настоящее время на развитие систем, в которых информация имеет характер изображений или видео. 

Вместе с тем, решение научных и инженерных задач при работе с визуальными данными требует особых усилий, опирающихся на знание специфических методов, поскольку традиционные методы обработки одномерных сигналов мало пригодны в этих случаях. В особой мере это проявляется при создании новых типов информационных систем, решающих такие проблемы, которые до сих пор в науке и технике не решались, и которые решаются сейчас благодаря использованию информации визуального характера. Эти проблемы решает теория компьютерного зрения.

Компьютерное зрение является динамично развивающимся направлением современной науки, востребованным в различных областях, начиная с интеллектуальных человеко-машинных интерфейсов, принятия решений роботами и заканчивая системами автоматического контроля на производстве. Неотъемлемой частью компьютерного зрения является распознавание образов, решающее задачу определения принадлежности входного изображения к одному из хранимых эталонных изображений объектов. При создании интеллектуальных систем также часто требуется отслеживать положение подвижных объектов в реальном времени на основе зрительной информации, полученной от видеокамеры. Располагая рядом последовательных по времени цифровых изображений или видеопотоком, можно выделить специальную информацию об объекте и затем использовать ее для обнаружения текущего положения объекта и отслеживания его перемещений. 

Для обеспечения отслеживания движения объектов  на видео, достаточно будет организовать последовательную передачу кадров видеопотока с камеры в программу для их обработки, с последующим использованием алгоритмов обнаружения для данной последовательности кадров.

К задачам компьютерного зрения можно отнести и проблему обнаружения на фото или видео человеческих лиц. Интерес к процессам отслеживания и распознавания лиц, всегда был значительным, особенно в связи с все возрастающими практическими потребностями: системы охраны, верификация кредитных карточек, криминалистическая экспертиза, телеконференции и т.д. Несмотря на ясность того житейского факта, что человек хорошо идентифицирует лица людей, совсем не очевидно, как научить этому компьютер, в том числе как декодировать и хранить цифровые изображения лиц. Так в последние десять лет или около того, распознавание лиц стало популярной областью исследований в компьютерном зрении и одним из самых успешных применений анализа изображений. Данная область привлекает не только исследователей в области информатики, а также неврологов и психологов.

Объектом изучения в рамках данного проекта будет библиотека алгоритмов компьютерного зрения OpenCV (open source computer vision) и методы обнаружения на видео движения и лиц.

В рамках проекта будет создана программа реализующая алгоритмы компьютерного зрения для поиска объектов в видеопотоке, получаемом с веб"=камеры. 
